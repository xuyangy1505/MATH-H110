\section{Vector Spaces}

%%%%%%%%%%%%%%%%%%%%%%%%%%%%%%%%%%%%%%%%%%%%%%%%%%%%%%%%%
%%%%%%%%%%%%%%%%% 1.1 PRELIMINARIES %%%%%%%%%%%%%%%%%%%%%
%%%%%%%%%%%%%%%%%%%%%%%%%%%%%%%%%%%%%%%%%%%%%%%%%%%%%%%%%
\subsection{Preliminaries}

%%%%%%%%%%%%%%%%%%%% 1.1.1 %%%%%%%%%%%%%%%%%%%%%%%%%%%%
\subsubsection{Sets}
\textbf{Definition} (Power Set). The \textbf{\textit{power set}} of a set $S$ is the collection, or \textbf{\textit{family}}, of subsets of $S$, denoted $P(S)$. E.g. the family of open intervals in $\mathbb{R}$ is a subset of $P(\mathbb{R})$.

\textbf{Definition} (Cartesian Product). For two sets $X$ and $Y$, $X\times Y:=\{(x,y)\mid x\in X, y\in Y\}$, also known as the \textbf{\textit{Cartesian product}} of the factors $X$ and $Y$. Denote $X^n$ to be the Cartesian product of a set with itself $n$ times.

\textbf{Definition} (Relation). A \textbf{\textit{relation}} between a set $X$ and $Y$ is any subset of $X\times Y$. A relation between a set and itself is said to be a relation \textit{on} $X$. 

\textbf{Definition} (Partial Order). A \textbf{\textit{partial order}} on set $S$ is a relation defined by $P\subseteq S\times S$ such that the following properties hold:
\begin{enumerate}
    \item \textbf{(antisymmetry)} $(a,b)\in P\wedge(b,a)\in P\implies a=b$.
    \item \textbf{(transitivity)} $(a,b)\in P\wedge(b,c)\in P\implies(a,c)\in P$.
    \item \textbf{(reflexivity)} $\forall a\in S$, $(a,a)\in P$.
\end{enumerate}
Typically, we write $a\leq b$ if $(a,b)\in P$. Furthermore we write $a<b$ if $a\leq b$ and $a\neq b$. So then the above properties become:
\begin{enumerate}
    \item \textbf{(antisymmetry)} $a\leq b\wedge b\leq a\implies a=b$.
    \item \textbf{(transitivity)} $a\leq b\wedge b\leq c\implies a\leq c$.
    \item \textbf{(reflexivity)} $a\leq a$.
\end{enumerate}
\textbf{Definition} (Poset). If there exists a partial order on $S$, then we say $S$ is a \textbf{\textit{partially-ordered-set}}, or \textbf{\textit{poset}}. If $a$ and $b$ are elements of a poset and $a\leq b$ or $b\leq a$, then we say that $a$ and $b$ are \textbf{\textit{comparable}}. Otherwise they are \textbf{\textit{incomparable}}.

\textbf{Definition} (Max \& Min). Suppose $P$ is a partially ordered set and $X\subseteq P$. Then $X$ inherits a partial order from $P$ (think subgraphs). For an element $u\in P$ s.t. $u\geq x$ for all $x\in X$, $u$ is an \textbf{\textit{upper bound}} of $X$. For an element $M\in X$ s.t. $M\geq x$ for all $x\in X$, $M$ is a \textbf{\textit{maximal}} element. Similarly, for an element $l\in P$ s.t. $l\leq x$ for all $x\in X$, $l$ is a \textbf{\textit{lower bound}} of $X$. For an element $m\in X$ s.t. $m\leq x$ for all $x\in X$, $m$ is a \textbf{\textit{minimal}} element.

\textbf{Definition} ("Toset"). A partial order in which every pair of elements is comparable is called a \textbf{\textit{total order}}. A set with a total order is known as a \textbf{\textit{totally ordered set}}. A \textbf{\textit{chain}} is a totally ordered subset of a partially ordered set

%%%%%%%%%%%%%%%%%% 1.1.2 %%%%%%%%%%%%%%%%%%%%%%%%%%
\subsubsection{Functions}
\textbf{Definition} (Function). A \textbf{\textit{function}} $f$ is defined to be a relation between sets $X$ and $Y$ s.t. for each $x\in X$ there exists exactly one $y\in Y$ such that the pair $(x,y)$ is included in the relation defined by $f$. The set $X$ is the \textbf{\textit{domain}} of $f$ and $Y$ is the \textbf{\textit{codomain}} of $f$. The symbol $\mapsto$ reads "maps to". Also, the collection of \textit{all} functions $f:X\to Y$ is denoted $\mathcal{F}(X,Y)$.

\textbf{Definition} (Composition). If $f:X\to Y$ and $g:Y\to Z$, then their \textbf{\textit{composition}} $g\circ f:X\to Z$ is defined by $(g\circ f)(x)=g(f(x))$. Know how to draw "commutative diagrams".

\textbf{Definition} (Power Functions). Given a function $f:X\to Y$, $P(f):P(X)\to P(Y)$ is equivalent to the function that maps a subset of $X$ to a subset of $Y$. Thus $P$ is in itself a function, $P: \mathcal{F}(X,Y)\to \mathcal{F}(P(X),P(Y))$. Also, $P^{-1}(f):P(Y)\to P(X)$ is the power function of $f$ that sends a subset in $Y$ to its preimage.

\textbf{Definition} (Image, Range, Preimage). Given a set $S\subseteq X$ and a function $f:X\to Y$, the \textbf{\textit{image}} is the set $f(S):=\{f(x)\mid x\in S\}$. If $S=X$, then $f(S)$ is the \textbf{\textit{range}} of $f$. The range is a subset of the codomain. For a set $T\subseteq Y$, the \textbf{\textit{preimage}} is the set $f^{-1}(T):=\{x\in X\mid f(x)\in T\}$. 

\textbf{Definition} (Function Inverse). Given a function $f:X\to Y$, $f$ is invertible if for each $y\in Y$, there exists a unique $x\in X$ s.t. $f^{-1}(y)=x$. In particular, it is invertible if its range is equivalent to its codomain and the preimage of each singleton in its range is a singleton in its domain.

\textbf{Definition} (Injective, Surjective, Bijective). A function $f:X\to Y$ always satisfies that if $x=y$, then $f(x)=f(y)$. $f$ is \textbf{\textit{injective}} is $f(x)=f(x')$ implies $x=x'$. Such a function is called a \textit{one-to-one} function. $f$ is \textbf{\textit{surjective}} if for all $y\in Y$, there exist an $x\in X$ s.t. $f(x)=y$. This property is also called \textit{onto}. If a function is both injective and surjective, then it is called \textbf{\textit{bijective}}, otherwise known as a \textbf{\textit{one-to-one correspondence}}.

\textbf{Proposition}. A function $f:X\to Y$ is invertible if and only if it is bijective.

\textbf{Definition} (Inclusion, Restriction, Extension). For a set $A\subseteq B$, the \textbf{\textit{inclusion}} function is defined by $\iota_{A,B}=\iota:A\to B$ s.t. for an $a\in A$, $\iota(a)=a$. 

On the other hand, for a set $S\subseteq A$, the \textbf{\textit{restriction}} of $f$ to $S$ for a function $f:A\to B$ is $f\mid_S:S\subseteq A\to B$ s.t. $f\mid_S(x)=f(x)$ for $x\in S$. 

For a set $T$ s.t. $B\subseteq T$, we can \textbf{\textit{extend}} the codomain of $f$ to $T$ by composition with $\iota:B\to T$. That is, $\iota\circ f:A\to T$ satisfies $\iota(f(x))=f(x)$ for all $x\in A$.

%%%%%%%%%%%%%%%%%%%%% 1.1.3 %%%%%%%%%%%%%%%%%%%%%%%%%%%%%
\subsubsection{Lists \& Sequences}
\textbf{Definition} (Lists \& Sequences). Let $X$ be a set and $s\in\mathbb{N}$. Then a \textbf{\textit{list}} $(x_1, x_2,..., x_n)$ is a function from $\{1,2,...,n\}\to X$ s.t. $i\mapsto x_i$. Order (permutation) matters and there can also be duplicity in lists. A \textbf{\textit{sequence}} is essentially an infinite list that takes $\mathbb{N}\to X$.

\textbf{Definition} (Countability). We say that a set is \textbf{\textit{countably infinite}} if there exists a one-to-one correspondence with $\mathbb{N}$. A set is \textbf{\textit{countable}} if it is either finite or countably infinite. An \textbf{\textit{uncountable}} set is one that is not countable. Two sets between which there exists a bijection are said to have the same cardinality.

\textbf{Definition} (Index Set). We say that a set $A$ indexes $S$ if $f:A\to S$ is a list or a sequence.

\textbf{Definition} (Dirac Delta Function). Let $S$ be a set. For $p\in S$, the \textbf{\textit{Dirac Delta function}} at $p$, also called the \textbf{\textit{indicator function}} at $p$, is the function $\delta_p:S\to \{0,1\}$ s.t. $\delta_p(s)=0$ if $s\neq p$ and $\delta_p(s)=1$ if $s=p$. The point $p$ is called the base point of $\delta_p$. Note also that there is a bijection between $S$ and $\delta_p$.

%%%%%%%%%%%%%%%%%%%%%%%% 1.1.4 %%%%%%%%%%%%%%%%%%%%%%%%%%
\subsubsection{Fields \& Complex Numbers}
An element of $\mathbb{F}(X\times X, X)$ is called a \textbf{\textit{binary operation}} on $X$. A binary operator $g:X\times X\to X$ is \textbf{\textit{commutative}} if $g(x_1, x_2)=g(x_2, x_1)$ for all $(x_1,x_2)\in X\times X$. A binary operation on $X$ is \textbf{\textit{associative}} if $g(x_1,g(x_2, x_3))=g(g(x_1,x_2),x_3)$ for all $x_1, x_2, x_3\in X$.

\textbf{Definition} (Group). A \textbf{\textit{group}} $G=(|G|,g)$ is the pairing of an underlying set $|G|$ with a binary operation $g:|G|\times|G|\to|G|$ satisfying the following three axioms:
\begin{enumerate}
    \item (associativity) $g$ is an associative operation on $|G|$.
    \item (identity) There exists an element $e\in|G|$ s.t. $g(e,a)=g(a,e)=a$ for all $a\in|G|$, known as the \textbf{\textit{identity}} element.
    \item (inverse) For every $a\in|G|$, there exists an element $a'\in|G|$ s.t. $g(a,a')=g(a',a)=e$.
\end{enumerate}
If $g$ is also a commutative operation on $|G|$, then $G$ is called an \textbf{\textit{abelian group}}.

\textbf{Definition} (Field). A \textbf{\textit{field}} $\mathbf{F}$ is a set $F$ equiped with two binary operations, "addition" and "multiplication", such that $(F,+)$ is an abelian group and $(F\setminus 0,\cdot)$ is an abelian group where "0" is the identity of $(F,+)$. Furthermore, for all $a, b, c\in F$, \textbf{\textit{distributivity of multiplication over addition}} holds:
\[
a\cdot(b+c)=a\cdot b + a\cdot c
\]
We write $\mathbf{F}$ as the triple $(F,+,\cdot)$, where $F$ is the underlying set of $\mathbf{F}$. In checking whether a set $F$ equipped with two binary operations is a field, we need to check whether addition and multiplication are "closed".

\textbf{Definition} (Characteristic). The \textbf{\textit{characteristic}} of a field is the smallest $n$ such that when "1" is "added" onto itself $n$ times, "0" is produced. If no such $n$ exists, then the characteristic is defined to be 0.

\textbf{Definition} (Polynomial). A \textbf{\textit{polynomial}} in one variable over a field $\mathbf{F}$ is defined to be a function $p:\mathbf{F}\to\mathbf{F}$ where
\[
p(z)=a_nz^n+a_{n-1}z^{n-1}+\ldots+a_1z+a_0
\]
for all $z\in\mathbf{F}$ and for some fixed $a_0,...,a_n\in\mathbf{F}$. Such a polynomial has similar constituent definitions to the classic definition of a polynomial.

\textbf{Definition} (Ordered Field). A field $\mathbf{F}$ together with a total order $\leq$ on $F$ is an \textbf{\textit{ordered field}} if the order satisfies the following properties:
\begin{enumerate}
    \item if $a\leq b$, then $a+c\leq b+c$;
    \item if $0\leq a$ and $0\leq b$, then $0\leq a\cdot b$.
\end{enumerate}
The sets $\mathbb{Q}$ and $\mathbb{R}$ are examples of ordered fields.


%%%%%%%%%%%%%%%%%%%%%%%%%%%%%%%%%%%%%%%%%%%%%%%%%
%%%%%%%% 1.2 DEFINITION OF VECTOR SPACE %%%%%%%%%
%%%%%%%%%%%%%%%%%%%%%%%%%%%%%%%%%%%%%%%%%%%%%%%%%
\subsection{Definition of a Vector Space}
A \textbf{\textit{vector space}} $V$ over a field $\mathbf{F}$ is a set $|V|$ together with the binary operation of \textbf{\textit{vector addition}}
\begin{align*}
    |V|\times|V|\to|V| \\
    (v,w)\mapsto v + w
\end{align*}
and \textbf{\textit{scalar multiplication}}
\begin{align*}
    \mathbf{F}\times|V|\to|V| \\
    (a,v)\mapsto av
\end{align*}
such that $(|V|,+)$ is an abelian group with the following axioms:
\begin{enumerate}
    \item Associativity of scalar multiplication
    \item Identity of scalar multiplication
    \item Distributivity over vector addition
    \item Distributivity over field addition
\end{enumerate}
Note that the elements of $\mathbf{F}$ are called \textit{scalars} whereas the elements of $|V|$ are called \textit{vectors}. So effectively, a vector space comes with 4 pieces of data: $(|V|, \mathbf{F}, \text{addition}, \text{scalar multiplication})$.



\newpage %%%%%% EXERCISES %%%%%%
\subsection{Exercises}
\textbf{1.1 Solution.} $x\leq y$ and $y\leq x$ if and only if $x=y$ in property (i), which satisfies the antisymmetry property. 

For $x,y,z\in S$, if at least one pair of them are equal to each other, then the statement of transitivity becomes trivial (if $x=y$ and $y\leq z$, then $x\leq z$; or if $x=y=z$ and $y\leq z$, then $x=z\rightarrow x\leq z$). Else if no pairs among $x,y,z$ hold an equality, then the property of transitivity is directly satisfied by property (ii). 

Finally, in the case of totality, $x\leq y$ or $y\leq x$ is the same as property (i), as both essentially say that every pair of elements in $S$ is comparable.

\textbf{1.2 Solution.} We construct a set $S$ and prove that it is a partial order on $\mathcal{F}(X,P)$. For two functions $f$ and $g$ in the function space $\mathcal{F}(X,P)$, $(f,g)\in S$ only if $f(x)\in P$ and $g(x)\in P$ for some $x\in X$. Now, we show that $S$ satisfies the three rules for being a partial order on function space $\mathcal{F}(X,P)$:
\begin{enumerate}
    \item (antisymmetry) Since $P$ is a partially ordered set and $f(x),g(x)\in P$, then if $f(x)\leq g(x)$ and $g(x)\leq f(x)$, $f(x)=g(x)$ by the antisymmetry of some partial order on $P$.
    \item (transitivity) If $f(x)\leq g(x)$ and $g(x)\leq h(x)$, then $f(x)\leq h(x)$ by the transitivity of some partial order on $P$. Therefore $(f,h)\in S$ as well.
    \item (reflexivity) For all functions $f$ in the function space, $(f,f)\in S$ since $f(x)=f(x)$ implies $f(x)\leq f(x)$.
\end{enumerate}
Thus, $S$ is a partial order on the function space $\mathcal{F}(X,P)$ determined by $f\leq g$ if $f(x)\leq g(x)$.

\textbf{1.3 Solution.} The Identity function takes $X$ back to itself, so $P(ID_X)=P(X)$. It also takes $P(X)$ back to itself, $P(X)$. Therefore $P(ID_X)=P(X)=ID_{P(X)}$. 

$P(g\circ f)$ takes a subset $S_X\subseteq X$ and sends it to a subset $S_Z\subseteq Z$, whereas $P(g)\circ P(f)$ takes the set $S_X$ to a set $S_Y\subseteq Y$ and then takes $S_Y$ to another set $T_Z\in Z$. For every $x\in S_X$, $P(g\circ f)$ sends $x$ to a $z=g(f(x))\in S_Z$. On the other hand, $P(g)\circ P(f)$ first sends $x$ to a $y=f(x)\in S_Y$, and then sends $y$ to $z=g(y)=g(f(x))\in T_Z$. But since both the LHS and RHS send $x\in S_X$ to $g(f(x))$, it must be true that $S_Z=T_Z$, and therefore the equality $P(g\circ f)=P(g)\circ P(f)$ holds.

\textbf{1.4 Solution.} (a) \boxed{\text{False.}} We provide a counter example. Let $S=\{1\}$, $X=\{1, 2\}$, $Y=\{3\}$, and $f(1)=f(2)=3$. Then $f(S)=\{3\}$, and $f^{-1}(\{3\})=\{1, 2\}$ since the preimage of $\{3\}$ includes the element 2 in addition to 1. Therefore $f^{-1}(f(S))\neq S$.

(b) \boxed{\text{True.}} Define function $f:X\to Y$. Since it is invertible, we have $f^{-1}:Y\to X$ s.t. every $x\in X$ maps to a unique $y\in Y$ and vice versa. In other words, $f$ is bijective. Then $P(f^{-1})$ takes a subset $S\subseteq Y$ and outputs a set $\{x\in X\mid f(x)\in S\}$, or the preimage of $S$ such that each $y$ maps to one unique $x$. On the other hand $P^{-1}(f)$ takes the subset $S$ and returns the preimage of $S$ by definition. Thus $P(f^{-1})$ and $P^{-1}(f)$ are the same function if $f^{-1}$ is invertible.

\textbf{1.5 Solution.} Let $S_Z\in P(Z)$. Then $P^{-1}(g\circ f)$ takes $S_Z$ to another set $S_X\in P(X)$. In other words, if $z\in S_Z$, then $P^{-1}(g\circ f)(z)=\{x\in X\mid g(f(x))=z\}$. Therefore $S_X$ is just the union of all the sets returned by inputting a $z$ into $P^{-1}(g\circ f)$.

On the other hand, $P^{-1}(g)$ takes some $z\in S_Z$ and returns a set $\{y\in Y\mid g(y)=z\}$. The union of all these returned sets forms another set $S_Y$, which we then input into $P^{-1}(f)$ by function composition. This then returns the final set $\{x\in X\mid \forall y\in S_Y, f(x)=y\}$. But this final set is equivalent to the set $\{x\in X\mid \forall z\in S_Z, g(f(x))=z\}$, thus $P^{-1}(g\circ f)=P^{-1}(f)\circ P^{-1}(g)$.

\textbf{1.6 Solution.} The preimage under $f$ of a circle of radius $r$ is the set $\{(r,\theta)\mid \theta\in\mathbb{R}\}$. \\
The preimage of the positive $x$-axis is the set $\{(r,2\pi n-2\pi)\mid r\in\mathbb{R}^+, n\in\mathbb{N}\}$. \\
The image of the rectangle $[0,r]\times[0,\pi/4]$ is the circular sector of radius $r$ between $0^{\circ}$ and $45^{\circ}$. \\
The restriction of $f$ to $L$ is $f\mid_L:L\subseteq\mathbb{R}^2\to \mathbb{R}^2$. Written as a surjective function, we have$f\mid_L:L\to X$ where $X$ is the set of points constituting the $x$-axis.

\textbf{1.7 Solution.} WLOG, let $a,b\in \mathbb{R}^+$ such that $a\geq b$. Then since $a*b=|a-b|=a-b=|b-a|=b*a$ and $a-b$ is always a nonnegative real number, we know that $*$ determines a commutative binary operator on $\mathbb{R}^+$

The inverse image of $x\in\mathbb{R}^+$ is the set $\{(a,b)\mid a,b\in\mathbb{R}^+, |a-b|=x\}$, or in other words the set of ordered pairs in $\mathbb{R}^+$ such that the absolute difference between the first coordinate and the second coordinate is equal to $x$.

Counterexample: $1*(2*3)=|1-|2-3||=0\neq 2=||1-2|-3|=(1*2)*3$. Thus $*$ is not associative.

\textbf{1.8 Solution.} In $\mathbb{Z}_2$, let the identity under addition be the element $0$. Then the inverse of $1$ under addition is $1$ and the inverse of $0$ under addition is $0$. $(\{0,1\},+)$ also satisfies associativity since we can effectively treat $+$ as normal addition in modulo 2, therefore $(\{0,1\},+)$ is an abelian group.

Now, we check that $(\{1\},\cdot)$ is an abelian group. Letting $1$ be the identity under $\cdot$, the inverse of $1$ is itself. Furthermore $1\cdot(1\cdot1)=(1\cdot1)\cdot1$, so $(\{1\},\cdot)$ is also an abelian group. Also, distributivity of $\cdot$ over $+$ is easily verifiable if we just treat the operations as normal addition and multiplication in modulo 2. Thus $\mathbb{Z}_2$ forms a field.

Finally, since $1+1=0$, we know that the characteristic of $\mathbb{Z}_2$ is 2.

\textbf{1.9 Solution.} Assume for sake of contradiction that $\mathbb{R}^2$ is a field when endowed with coordinate-wise addition and multiplication. Then the element $(0,0)$ must be the identity under addition, and $(1,1)$ must be the identity under multiplication. For $\mathbb{R}^2$ to be a field, every element that is not $(0,0)$ must have a multiplicative inverse. However, the element $(0,1)$ cannot have a multiplicative inverse as there exists no such $x\in\mathbb{R}$ such that $x\cdot 0 = 1$ in the first coordinate. $\Rightarrow\Leftarrow$

Thus $\mathbb{R}^2$ cannot be a field when endowed with coordinate-wise addition and multiplication.

\textbf{1.10 Solution.} First we show that $(\mathbb{C},+)$ is an abelian group. Associativity and commutativity are easily satisfied if we look at the real part and imaginary part separately, treating it the same as regular addition of real scalars, which is both associative and commutative. The identity is $0+0i=0$, and for all $z\in\mathbb{C}$, the inverse of $z=a+bi$ is just $-z=-a-bi$. Therefore $(\mathbb{C},+)$ is an abelian group.

Now we show that $(\mathbb{C}\setminus0,\cdot)$ is also an abelian group. For all $a,b,c,d,e,f\in\mathbb{R}$, $((a+bi)\cdot(c+di))\cdot(e+fi)=(ace-bde-adf-bcf)+(acf-bdf+bce+ade)i=(a+bi)\cdot((c+di)\cdot(e+fi))$, so $\cdot$ is associative on $\mathbb{C}$. Also, $(a+bi)\cdot(c+di) = (ac-bd)+(ad+bc)i = (c+di)\cdot(a+bi)$, so $\cdot$ is also commutative. The identity is the element $1+0i=1$. For every $z=a+bi\in\mathbb{C}\setminus0$, the inverse of $z$ can be verified to be $z^{-1}=(\frac{a}{a^2+b^2})+(\frac{-b}{a^2+b^2})i$. Therefore $(\mathbb{C}\setminus0,\cdot)$ is an abelian group.

Finally, we show that distributivity of multiplication over addition holds as well for all $a,b,c,d,e,f\in\mathbb{C}$:
\begin{align*}
(a+bi)\cdot((c+di)+(e+fi)) &= (ac+ae-bd-bf)+(ad+af+bc+be)i \\
                           &= (a+bi)\cdot(c+di)+(a+bi)\cdot(e+fi).
\end{align*}

Thus the set $\mathbb{C}$ of complex numbers does indeed form a field.

\textbf{1.11 Solution.} First we note that $1\neq0$ since $1$ is defined as the multiplicative identity for the group $(F\setminus 0, \cdot)$ which doesn't include $0$. 

Now, we apply distributivity: for any $a\in\mathbf{F}$, $a\cdot(0+0)=a\cdot0=a\cdot0+a\cdot0$. Adding the additive inverse to both sides, we get $a\cdot0=0$. Then we also have $0=-1\cdot0=-1\cdot(1+(-1))=-1+(-1)^2$. Therefore $(-1)^2=1$. 

Now, we assume for sake of contradiction that $1<0$. Then since $\mathbf{F}$ is an ordered field, we can add $-1$ to both sides to get $0<-1$. Then by the second condition of an ordered field, we know that $0<(-1)(-1)=1$, a contradiction. Thus, since $1\neq0$ and $1\nless0$, it must be true that $1>0$.

Next, in order to prove $a^2\geq 0$ for all $a\in\mathbf{F}$, we first show that $a^2=(-a)(-a)$. Earlier we showed that $a\cdot 0 = 0$. Then using distributivity, we have $0=a\cdot (-1+1)=a\cdot(-1)+a$. Adding $-a$ to both sides, we get $a\cdot(-1)=-a$. Then it follows that $(-a)(-a)=(-1)(a)(-1)(a)=(-1)(-1)(a)(a)=a^2$, since we also proved earlier that $(-1)(-1)=1$. 

Now, if $a\geq0$, then by the second property of an ordered field, $a^2=(a)(a)\geq0$ and we are done. Instead, if $a\leq0$, then by adding $-a$ to both sides we get $-a\geq0$. Then $(-a)(-a)\geq0$, but since $(-a)(-a)=a^2$, it implies that $a^2\geq0$ too. Thus we are done and $a^2\geq0$ for all $a\in\mathbf{F}$.

Finally, from our findings we deduce that $\mathbb{C}$ is not an ordered field by analyzing $i$. If $i\geq0$, then $-1=(i)(i)\geq0$, a contradiction. On the other hand, if $i\leq0$, then $-i\geq0$, and $-1=(-i)(-i)\geq0$, also a contradiction. Therefore the complex numbers cannot form an ordered field.

*\textit{All \textbf{bolded} numbers represent vectors, non-bolded represent scalars}

\textbf{1.12 Solution.} We start with the LHS:

\begin{tabular}{rll}
0v &= 0v + \textbf{0} &\text{Identity element in vector addition} \\
   &= 0v + (v+(-v))   &\text{Inverse elements in vector addition} \\
   &= (0v+v)+(-v)     &\text{Associativity of vector addition} \\
   &= ((0+1)v)+(-v)   &\text{Distributivity of field addition} \\ 
   &= ((1)v) + (-v) &\text{Identity element in field addition} \\
   &= v+(-v)          &\text{Identity element in scalar multiplication} \\ &= \textbf{0}            &\text{Inverse elements of vector addition}
\end{tabular}

\textbf{1.13 Solution.} From our result in \textbf{1.12}, we have that $\textbf{0} = 0v$. We also have that $0v = (1+(-1))v$ by inverse elements in field addition. Then by distributivity of field addition, we have $0v = 1v+(-1)v$. So we have $\textbf{0} = v + (-1)v$. Adding the vector additive inverse of $v$ to both sides, we get $\textbf{0}+(-v)=(-1)v$, or just $-v=(-1)v$, by the identity element in vector addition.

Then $-(-v)$ is equivalent to $-(-1(v))$, which is equivalent to $(-1)(-1(v))$. By associativity of scalar multiplication, that is equivlanet to $(-1)^2v$. In exercise \textbf{1.11} we proved that $(-1)^2=1$, so finally we have $(-1)^2v=v$. Thus $-(-v)=v$.

\textbf{1.14 Solution.} First we prove the "if" direction. If $a=0$, then $0v=\textbf{0}$ is true by the result from exercise \textbf{1.12}. If $v=\textbf{0}$, then $a\textbf{0}=\textbf{0}$ by property (b). Thus we are done proving the "if" direction.

Now we prove the "only if" direction. Let $a$ be a scalar in $\mathbf{F}$ and $v$ be a vector in $|V|$. Also let $av=\textbf{0}$. We proceed by cases. Case 1: if $a=0$, then we are done. Case 2: if $a\neq0$, then the multiplicative inverse $a^{-1}$ of $a$ must exist in $\mathbf{F}$. Multiplying both sides of $av=0$ by $a^{-1}$, we get $(a^{-1}a)v=a^{-1}\textbf{0}$ using associativity of scalar multiplication. The LHS simplifies to $v$, and the RHS simplifies to $\textbf{0}$ by property (b). Therefore $v=\textbf{0}$ and the "only if" direction is also true. Thus $av=\textbf{0}$ if and only if $a=0$ or $v=\textbf{0}$.
\newpage
\textbf{1.14 Solution.} First, note that $\mathcal{C}([a,b],\mathbb{R})$ is nonempty. Since we can assume that the sum of two continuous functions is continuous, then we know that for two functions $c_1, c_2\in\mathcal{C}([a,b], \mathbb{R})$, the function $c_1+c_2$ must also be in $\mathcal{C}([a,b], \mathbb{R})$. Furthermore, since we can assume that a scalar multiple of a continuous function is also continuous, then we know that for any scalar $a$ and any function $c\in\mathcal{C}([a,b],\mathbb{R})$, the function $ac$ is also in $\mathcal{C}([a,b],\mathbb{R})$.

\textbf{1.15 Solution.} No, the set is not closed under addition and scalar multiplication. That is, for a polynomial $ax^2+bx+1$, $(ax^2+bx+1)+(ax^2+bx+1)= 2ax^2+2bx+2$ cannot be in the set because the constant term is 2, whereas the set only contains polynomials where the constant term is 1.

\textbf{1.16 Solution.} First, note that the set of all polynomials with degree $\leq n$ is nonempty. Next, note that any polynomial with degree $\leq n$ must also be in the set of all polynomials over $\mathbb{F}$. For two polynomials $p_1(x)$ and $p_2(x)$ of degree $\leq n$, the polynomial $p_1(x)+p_2(x)$ is also a polynomial of degree $\leq n$, since addition of corresponding coefficients is closed in the field $\mathbb{F}$. Finally, for a polynomial $p(x)$ of degree $\leq n$ and a scalar $a\in\mathbb{F}$, the polynomial $a(p(x))$ is also a polynomial of degree $\leq n$, since multiplication of the coefficients by a scalar is also closed in the field $\mathbb{F}$. Therefore it is indeed a subspace.

Yes, any set of polynomials under the condition that only a predetermined set of powers have terms with nonzero coefficients. For example, the set of polynomials $\{ax^8+bx^5+cx^2+d\mid a,b,c,d\in\mathbb{F}\}$ is a subspace of $\mathcal{P}(\mathbb{F})$.

\textbf{1.17 Solution.}
(a) Let $T$ and $U$ be two subspaces of $V$ over the field $\mathbb{F}$. Then since $T$ and $U$ are closed under scalar multiplication, (if we let the scalar be 0) the vector \textbf{0} in $V$ must also be in both $T$ and $U$, hence $\textbf{0}\in T\cap U$. This means that $T\cap U$ is nonempty. Now, suppose we have vectors $v_1,v_2\in T\cap U$, then $v_1$, $v_2$, and $v_1+v_2$ are all in $T$, and similarly $v_1$, $v_2$, $v_1+v_2$ are all in $U$ as well. Thus $v_1+v_2\in T\cap U$ and $T\cap U$ is closed under addition. Finally, suppose we have a vector $v\in T\cap U$ and a scalar $a\in\mathbb{F}$. Then it follows that since $v\in T$, $av\in T$, and similarly since $v\in U$, $av\in U$. Thus $av\in T\cap U$ and $T\cap U$ is closed under scalar multiplication. Therefore $T\cap U$ is indeed a subspace of $V$.

(b) The $x$-axis and $y$-axis are both subspaces of the $xy$-plane over the field of real numbers $\mathbb{R}$. However, their union is not a valid subspace of the $xy$-plane since it is not closed under addition.

(c) Assume for sake of contradiction that there is a subspace $S$ of $\mathbb{F}$ other than $\mathbb{F}$ and \textbf{0}. Then for vectors $a,b\in\mathbb{F}$, $a+b$ must be in $S$. Also, for a "scalar" $a\in\mathbb{F}$ and a "vector" $b\in\mathbb{F}$, the $ab$ must also be in $S$. But since all of these ``scalars'' and ``vectors'' are elements of the same set $\mathbb{F}$, the set $S$ is equivalent to the underlying set of $\mathbb{F}$, since $\mathbb{F}$ is closed under addition and multiplication operations. Thus the only subspaces of $\mathbb{F}$ are $\mathbb{F}$ itself and \textbf{0}, which is a subspace of all vector spaces.

\textbf{1.18 Solution.} First, $span(X)$ is always nonempty, since if $X$ were empty, then $span(X)$ would be the subspace $\{0\}$. Now, suppose we have vectors $v_1, v_2\in X$, then by the definition of $span(X)$, $v_1,v_2\in span(X)$ and therefore $v_1+v_2$ must also exist in $span(X)$. Thus $span(X)$ is closed under addition. Furthermore, suppose we have a vector $v\in X$ and a scalar $a\in\mathbb{F}$, then once again by the definition of $span(X)$, $v\in span(X)$ and so $av$ must also exist in $span(X)$. Thus $span(X)$ is also closed under scalar multiplication and is therefore a subspace of $V$.

\textbf{1.19 Solution.} We will show that $S(V)$ is an abelian group without inverses under the $\cap$ ("intersection") operation. 

Set intersection is associative. In particular, if we have an element $x$ in either $S_1\cap(S_2\cap S_3)$ or $(S_1\cap S_2)\cap S_3$, then it is true that $x$ is also in the other since both statements are the same as saying $x$ is in $S_1$, $S_2$, \textit{and} $S_3$.

Set intersection is also commutative. In particular, if an element $x$ is in either $S_1\cap S_2$ or $S_2\cap S_1$, then it is true that $x$ is also in the other since both statements are the same as saying $x$ is in $S_1$ \textit{and} $S_2$.

Finally, for every subspace $U\in S(V)$, the identity element under $\cap$ is $|V|$, since the intersection of original vector space with any subspace $U$ is just $U$ itself.

\textbf{1.20 Solution.} First, let $v_1+U=v_2+U$. Then if we add the additive inverse of $v_2$ to both sides, we get $v_1+U-v_2=v_2+U-v_2$. Since vector addition is commutative, we then have $v_1-v_2+U=U$. Then, since $U$ is a subspace, it must be closed under addition, and so  $v_1-v_2+\textbf{0}\in (v_1-v_2+U)=U$, therefore the vector $v_1-v_2$ must be in $U$.

Now, we prove the converse. Let $v_1-v_2\in U$. Then, since $U$ is a subspace, it must be closed under addition. Then the set $\{v_1-v_2+u\mid u\in U\}$ is equivalent to $\{u\mid u\in U\}$, so the affine subset $v_1-v_2+U$ is equivalent to the affine subset $U$. Finally, if we "add" $v_2$, which is the additive inverse of $-v_2$, we get that $v_2+U$ is an equivalent affine subset to $v_1+U$ by the commutativity of vector addition. 

Thus, $v_1+U=v_2+U$ if and only if $v_1-v_2\in U$.

\textbf{1.21 Solution.} First, let $v+U$ be a subspace of $V$. Then $v+U=\{v+u\mid u\in U\}$. Since we showed earlier that \textbf{0} is an element of every vector subspace, we let $u=\textbf{0}$ to get that $v\in v+U$. Then because $v+U$ is a subspace it must be closed under addition, and therefore $v+v\in\{v+u\mid u\in U\}$. But this can only happen if we let $u=v$, so therefore $v\in U$.

Now, we prove the converse. Let $v\in U$. Then since $U$ is closed under addition, we take the set $v+U=\{v+u\mid u\in U\}$ and deduce that it is the same set as $U$, which is a subspace. Therefore $v+U$ is a subspace of $V$, since $U$ is a subspace of $V$.

\textbf{1.22 Solution.} First, we know that $(\{v+U\mid v\in V\}, ``+")$ is an abelian group because the properties of commutativity and associativity carry over. Also, the identity element under ``+'' is just $U$, and the inverse of each $v+U$ is $-v+U$.

Associativity of scalar multiplication follows from $a(b(v+U))=abv+U=(ab)(v+U)$. The scalar identity is still the element $1\in\mathbb{F}$, such that $1(v+U)=v+U$.

Distributivity over vector addition also holds, since $a((v_1+U)``+"(v_2+U))=a(v_1+v_2+U)=a(v_1+v_2)+U=av_1+av_2+U=(av_1+U)``+"(av_2+U)=a(v_1+U)``+"a(v_2+U)$.

Finally, distributivity over field addition also holds, since $(a+b)(v+U)=(a+b)v+U=av+bv+U=(av+U)``+"(bv+U)=a(v+U)``+"b(v+U)$.

Thus $V\setminus U$ is a vector space since it satisfies all the vector space axioms.

\textbf{1.23 Solution.} Let $S$ be an affine subset of $V$. Then we can treat each of the $\lambda_i$'s as a "weight", such that the sum of the product of the vectors will still be in $S$. I got stuck here, but I have an idea where we prove that $\lambda_1x_1+\lambda_2x_2$ is in $S$ and then use induction to generalize to a list of $m$ scalars and $m$ vectors.

\textbf{1.24 Solution.} (a) If we take vector addition to be coordinate-wise addition, then it is a homomorphism: $F(x+x',y+y',z+z')=(z+z',y+y')=(z,y)+(z',y')=F(x,y,z)+F(x',y',z')$.

(b) Not a homomorphism: $F(x+x',y+y',z+z')=(x+x'+1,y+y',z+z'+1)\neq(x+x'+2,y+y',z+z'+2)=(x+1,y,z+1)+(x'+1,y',z'+1)=F(x,y,z)+F(x',y',z')$.

(c) Not a homomorphism:
$F(x+x',y+y')=(xy+xy'+x'y+x'y',0)\neq(xy+x'y',0)=(xy,0)+(x'y',0)=F(x,y)+F(x',y')$.

(d) Is a homomorphism:
$F(x+x',y+y')=(2x+2x'+3y+3y',x+x'-y-y')=(2x+3y,x-y)+(2x'+3y',x'-y')=F(x,y)+F(x',y')$.

\textbf{1.25 Solution.} We showed earlier that if $T\in Hom(U,V)$, then $\lambda T\in Hom(U,V)$ since $Hom(U,V)$ is a subspace of $\mathcal{F}(U,V)$. Then we only need to show that there exists a linear mapping $(\lambda T)^{-1}$ that takes $V$ back to $U$. To do this, we make use of the fact that $T$ is an isomorphism, and therefore there $T^{-1}$ is a homomorphism that satisfies additivity and homogeneity.

If we define $(\lambda T)^{-1}$ as $\lambda(T^{-1})$, then it satisfies the homomorphic properties of additivity and homogeneity:

In particular, we have $(\lambda T)^{-1}(v+v')=\lambda(T^{-1}(v)+T^{-1}(v'))=(\lambda T)^{-1}(v)+(\lambda T)^{-1}(v')$. So additivity holds.

Additionally, for some scalar $a\in\mathbb{F}$, we have $(\lambda T)^{-1}(av)=\lambda(T^{-1}(av))=\lambda(aT^{-1}(v))=a(\lambda T^{-1}(v))$. So homogeneity holds as well.

Therefore $\lambda T$ must also be an isomorphism.

\textbf{1.26 Solution.} In $\mathbb{C}_\mathbb{R}$ scalar multiplication essentially "scales" vectors in $\mathbb{C}$ by a real $a\in\mathbb{R}$. On the other hand, in $\mathbb{C}_\mathbb{C}$, scalar multiplication is actually "vector multiplication", in the sense that resulting vector is the result of multiplying two complex numbers.

First we show that the map $M:\mathbb{C}_\mathbb{R}\to \mathbb{R}^2$ is bijective. Since $a+bi$ maps to $(a,b)$, we can easily define an inverse mapping $(a,b)\to a+bi$. Therefore there is a one-to-one correspondence between $\mathbb{C}_\mathbb{R}$ and $\mathbb{R}^2$. Now all we have to show is that $M$ is a homomorphism. 

Since $M((a+bi)+(a'+b'i))=(a+a',b+b')=(a,b)+(a',b')=M(a+bi)+M(a'+b'i)$, $M$ satisfies additivity. Furthermore, since $M(\lambda(a+bi))=M(\lambda a+\lambda bi)=(\lambda a, \lambda b) = \lambda (a, b)=\lambda M(a+bi)$ for $\lambda\in\mathbb{R}$, $M$ also satisfies homogeneity. Therefore $M$ is an isomorphism.

Now, we will show that $M':\mathbb{C}_\mathbb{C}\to\mathbb{R}^2$ satisfies additivity but not homogeneity.

Since $M'((a+bi)+(a'+b'i))=(a+a',b+b')=(a,b)+(a',b')=M'(a+bi)+M'(a'+b'i)$, $M'$ satisfies additivity. If $z=c+di\in\mathbb{F}$, then we have $M'(z(a+bi))=(ac-bd, ad+bc)$. However, $zM'(a+bi)$ is invalid because we cannot multiply a coordinate pair $(a,b)\in\mathbb{R}^2$ by a complex scalar $z\in\mathbb{C}$, so $M'$ doesn't satisfy homogeneity.

\textbf{1.27 Solution.} %Call the complex conjugation map $M$. Then $M((a+bi)+(a'+b'i))=M((a+a')+(b+b')i)=(a+a')-(b+b')i=(a-bi)+(a'-b'i)=M(a+bi)+M(a'+b'i)$, so $M$ satisfies additivity. \textbf{Note: I got sort of stuck here... } Now, homogeneity can only hold if we define $M(z(a+bi))=M(z)M(a+bi)$ for all scalars $z\in\mathbb{C}$. That is, let $z=c+di$ be a scalar, then $M((c+di)(a+bi))=M((ac-bd)+(ad+bc)i)=(ac-bd)-(ad+bc)i=(c-di)(a-bi)=M(z)M(a+bi)$ (Notice that it does not equal to $z(M(a+bi))$!).
For $z=a+bi\in\mathbb{C}$, $z\cdot\overline{z}=a^2+b^2$.

\textbf{1.28 Solution.} First, we show that for two complex numbers $z,w$,  $\overline{z+w}=\overline{z}+\overline{w}$. This is fairly easy to show, as if we let $z=a+bi$ and $w=c+di$, then we have $\overline{z+w}=\overline{a+bi+c+di}= a+c-(b+d)i= (a-bi)+(c-di)=\overline{a+bi}+\overline{c+di}$.

Next, we show that $\overline{cz}=c\overline{z}$ for some scalar $c$. This is also fairly easy to show, as we have $\overline{c(z)}=\overline{ca+cbi}=ca-cbi=c(a-bi)=c\overline{z}$.

Finally, we show that $\overline{zw}=\overline{z}\cdot\overline{w}$. This is just algebra, and we get $\overline{zw}=\overline{(ac-bd)+(ad+bc)i}=(ac-bd)-(ad+bc)i=(a-bi)(c-di)=\overline{a+bi}\cdot\overline{c+di}$.

Now, let $z$ be a complex root to a polynomial $c_nx^n+c_{n-1}x^{n-1}+\ldots+c_0$ for scalars $c_i$. Then $c_nz^n+c_{n-1}z^{n-1}+\ldots+c_0=0$. Taking the conjugate of the entire expression and using what we proved above, we get:
\begin{align*}
0 &= \overline{c_nz^n+c_{n-1}z^{n-1}+\ldots+c_0} \\ &= \overline{c_nz^n}+\overline{c_{n-1}z^{n-1}}+\ldots+\overline{c_0} \\ &= c_n\overline{z^n}+c_{n-1}\overline{z^{n-1}}+\ldots+c_0 \\ &= c_n\overline{z}^n+c_{n-1}\overline{z}^{n-1}+\ldots+c_0
\end{align*}

Thus the conjugate $\overline{z}$ must also be a root of the polynomial.

\textbf{1.29 Solution.} Let $v,v'$ be two vectors in $ker(T)$ and $a$ a scalar in $\mathbb{F}$. Then by additivity of $T$, we have that $T(v+v')=T(v)+T(v')=\textbf{0}+\textbf{0}=\textbf{0}$, which means the vector $v+v'\in ker(T)$ and so $ker(T)$ is closed under addition. Furthermore by the homogeneity of $T$, we have that $T(av)=aT(v)=a\textbf{0}=\textbf{0}$, which means the vector $av\in ker(T)$ and so $ker(T)$ is closed under scalar multiplication as well. Thus $ker(T)$ is a subspace of $V$. 

If $U$ is a subspace of $V$, then if we have vectors $u,u'\in U$ such that $T(u),T(u')\in ran(T)$, then by additivity of $T$ we know that $T(u+u')\in ran(T)$ so therefore $T(u)+T(u')\in ran(T)$. Furthermore, if we have a scalar $a\in\mathbb{F}$, then since $U$ is closed under scalar multiplication and by the homogeneity of $T$, we have that $T(au)\in ran(T)$ so therefore $aT(u)\in ran(T)$. Thus $ran(T)$ is closed under addition and scalar multiplication and is a subspace of $W$.

\textbf{1.30 Solution.} Omitted.

\textbf{1.31 Solution.} First we show that for any linear map $T:V\to W$, $T(\textbf{0})=\textbf{0}$. Due to homogeneity of $T$, we have $T(\textbf{0})=T(0\cdot v)=0\cdot T(v)=\textbf{0}$, for $v\in V$.

Now, for a vector $u\in ker(T)$, we have 
\begin{align*}
w &= T(v_0) \\
    &= T(v_0 + \textbf{0}) \\
    &= T(v_0) + T(0) \\
    &= T(v_0) + \textbf{0} \\
    &= T(v_0) + T(u) \\
    &= T(v_0+u),
\end{align*}
so then we have $T^{-1}(w)=v_0+u$. This is equivalent to the affine subset $v_0+ker(T)=\{v_0+u\mid u\in ker(T)\}$. Therefore any solution to the equation $T(v)=w$ is of the form $v_0+u$ for $u\in ker(T)$.

\textbf{1.32 Solution.} The matrix
$\left(\begin{tabular}{cccc}
    1 & 0 & 0 & 0 \\
    0 & 0 & 0 & 1
\end{tabular}\right)$
takes $(x_1,x_2,x_3,x_4)\mapsto(x_1,x_4)$

\textbf{1.33 Solution.} The matrix associated with this homomorphism is $aI$, or:
\begin{center}$\left(\begin{tabular}{cccc}
    a & 0 & $\ldots$ & 0 \\
    0 & a & $\ldots$ & 0 \\
    $\vdots$ & $\vdots$ &  & $\vdots$ \\
    0 & 0 & $\ldots$ & a
\end{tabular}\right)$\end{center}

\textbf{1.34 Solution.} (a) Subtracting $I$ from both sides, we get $A^2+2A=-I$. Since matrix multiplication is distributive, we get $A(A+2I)=-I$. Now, multiplying both sides by $-I$, we get $A(-A-2I)=I$, therefore the inverse of $A$ exists and is $-A-2I$.

(b) $A^n$ for $n\in\mathbb{N}$ is the 2 by 2 matrix $\left(\begin{tabular}{cc}
    1 & $n\cdot a$ \\
    0 & 1
    \end{tabular}\right)$. The inverse of $A$ is the 2 by 2 matrix $\left(\begin{tabular}{cc}
    1 & -a \\
    0 & 1
    \end{tabular}\right)$.
    
\textbf{1.35 Solution.} If a matrix $A$ is invertible, then there exists a matrix $B$ such that $AB=BA=I$. Then for the linear map that it determines $T:V\to W$, if we have a vector $v\in V$, $Av=w$ for some $w\in W$, but since $Bw=BAv=Iv=v$, the inverse $T^{-1}$ is just the matrix $B$, thus the linear mapping $T$ is invertible.

Now, assume $T:V\to W$ is an invertible linear mapping. Then there exists an inverse linear mapping $T^{-1}:W\to V$. Say $T$ can be represented by the matrix $A$. Then since $T$ is invertible, let $B$ be the matrix that represents $T^{-1}$, then we have that $v=T^{-1}(T(v))=BAv$. Thus $A$ is invertible since there exists a matrix $B$ such that $BA=I$.

\textbf{1.36 Solution.} The inverse of $A$ is the n by n matrix:
\begin{center}
$\left(
\begin{tabular}{cccc}
    $\frac{1}{a_1}$ & 0 & $\ldots$ & 0 \\
    0 & $\frac{1}{a_2}$ & $\ldots$ & 0 \\
    $\vdots$ & $\vdots$ &  & $\vdots$ \\
    0 & 0 & $\ldots$ & $\frac{1}{a_n}$
\end{tabular}
\right)$
\end{center}
\newpage
\textbf{1.37 Solution.}
