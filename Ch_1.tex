\section{Vector Spaces}
\subsection{Preliminaries}
\subsubsection{Sets}
%%% POWER SET %%%
\textbf{Definition} (Power Set). The \textbf{\textit{power set}} of a set $S$ is the collection, or \textbf{\textit{family}}, of subsets of $S$, denoted $P(S)$. E.g. the family of open intervals in $\mathbb{R}$ is a subset of $P(\mathbb{R})$.

%%% CARTESIAN PRODUCT %%%
\textbf{Definition} (Cartesian Product). For two sets $X$ and $Y$, $X\times Y:=\{(x,y)\mid x\in X, y\in Y\}$, also known as the \textbf{\textit{Cartesian product}} of the factors $X$ and $Y$. Denote $X^n$ to be the Cartesian product of a set with itself $n$ times.

%%% POSETS %%% 
\textbf{Definition} (Relation). A \textbf{\textit{relation}} between a set $X$ and $Y$ is any subset of $X\times Y$. A relation between a set and itself is said to be a relation \textit{on} $X$. 

\textbf{Definition} (Partial Order). A \textbf{\textit{partial order}} on set $S$ is a relation defined by $P\subseteq S\times S$ such that the following properties hold:
\begin{enumerate}
    \item \textbf{(antisymmetry)} $(a,b)\in P\wedge(b,a)\in P\implies a=b$.
    \item \textbf{(transitivity)} $(a,b)\in P\wedge(b,c)\in P\implies(a,c)\in P$.
    \item \textbf{(reflexivity)} $\forall a\in S$, $(a,a)\in P$.
\end{enumerate}
Typically, we write $a\leq b$ if $(a,b)\in P$. Furthermore we write $a<b$ if $a\leq b$ and $a\neq b$. So then the above properties become:
\begin{enumerate}
    \item \textbf{(antisymmetry)} $a\leq b\wedge b\leq a\implies a=b$.
    \item \textbf{(transitivity)} $a\leq b\wedge b\leq c\implies a\leq c$.
    \item \textbf{(reflexivity)} $a\leq a$.
\end{enumerate}
\textbf{Definition} (Poset). If there exists a partial order on $S$, then we say $S$ is a \textbf{\textit{partially-ordered-set}}, or \textbf{\textit{poset}}. If $a$ and $b$ are elements of a poset and $a\leq b$ or $b\leq a$, then we say that $a$ and $b$ are \textbf{\textit{comparable}}. Otherwise they are \textbf{\textit{incomparable}}.

\textbf{Definition} (Max \& Min). Suppose $P$ is a partially ordered set and $X\subseteq P$. Then $X$ inherits a partial order from $P$ (think subgraphs). For an element $u\in P$ s.t. $u\geq x$ for all $x\in X$, $u$ is an \textbf{\textit{upper bound}} of $X$. For an element $M\in X$ s.t. $M\geq x$ for all $x\in X$, $M$ is a \textbf{\textit{maximal}} element. Similarly, for an element $l\in P$ s.t. $l\leq x$ for all $x\in X$, $l$ is a \textbf{\textit{lower bound}} of $X$. For an element $m\in X$ s.t. $m\leq x$ for all $x\in X$, $m$ is a \textbf{\textit{minimal}} element.

\textbf{Definition} ("Toset"). A partial order in which every pair of elements is comparable is called a \textbf{\textit{total order}}. A set with a total order is known as a \textbf{\textit{totally ordered set}}. A \textbf{\textit{chain}} is a totally ordered subset of a partially ordered set

\subsubsection{Functions}
\textbf{Definition} (Function). A \textbf{\textit{function}} $f$ is defined to be a relation between sets $X$ and $Y$ s.t. for each $x\in X$ there exists exactly one $y\in Y$ such that the pair $(x,y)$ is included in the relation defined by $f$. The set $X$ is the \textbf{\textit{domain}} of $f$ and $Y$ is the \textbf{\textit{codomain}} of $f$. The symbol $\mapsto$ reads "maps to". Also, the collection of \textit{all} functions $f:X\to Y$ is denoted $\mathcal{F}(X,Y)$.

\textbf{Definition} (Composition). If $f:X\to Y$ and $g:Y\to Z$, then their \textbf{\textit{composition}} $g\circ f:X\to Z$ is defined by $(g\circ f)(x)=g(f(x))$. Know how to draw "commutative diagrams".

\textbf{Definition} (Power Functions). Given a function $f:X\to Y$, $P(f):P(X)\to P(Y)$ is equivalent to the function that maps a subset of $X$ to a subset of $Y$. Thus $P$ is in itself a function, $P: \mathcal{F}(X,Y)\to \mathcal{F}(P(X),P(Y))$. Also, $P^{-1}(f):P(Y)\to P(X)$ is the power function of $f$ that sends a subset in $Y$ to its preimage.

\textbf{Definition} (Image, Range, Preimage). Given a set $S\subseteq X$ and a function $f:X\to Y$, the \textbf{\textit{image}} is the set $f(S):=\{f(x)\mid x\in X\}$. If $S=X$, then $f(S)$ is the \textbf{\textit{range}} of $f$. The range is a subset of the codomain. For a set $T\subseteq Y$, the \textbf{\textit{preimage}} is the set $f^{-1}(T):=\{x\in X\mid f(x)\in T\}$. 

\textbf{Definition} (Function Inverse). Given a function $f:X\to Y$, $f$ is invertible if for each $y\in Y$, there exists a unique $x\in X$ s.t. $f^{-1}(x)=y$. In particular, it is invertible if its range is equivalent to its codomain and the preimage of each singleton in its range is a singleton in its domain.









\newpage %%%%%% EXERCISES %%%%%%
\subsection{Exercises}
\textbf{1.1 Solution.} $x\leq y$ and $y\leq x$ if and only if $x=y$ in property (i), which satisfies the antisymmetry property. 

For $x,y,z\in S$, if at least one pair of them are equal to each other, then the statement of transitivity becomes trivial (if $x=y$ and $y\leq z$, then $x\leq z$; or if $x=y=z$ and $y\leq z$, then $x=z\rightarrow x\leq z$). Else if no pairs among $x,y,z$ hold an equality, then the property of transitivity is directly satisfied by property (ii). 

Finally, in the case of totality, $x\leq y$ or $y\leq x$ is the same as property (i), as both essentially say that every pair of elements in $S$ is comparable.

\textbf{1.2 Solution.} We construct a set $S$ and prove that it is a partial order on $\mathcal{F}(X,P)$. For two functions $f$ and $g$ in the function space $\mathcal{F}(X,P)$, $(f,g)\in S$ only if $f(x)\in P$ and $g(x)\in P$ for some $x\in X$. Now, we show that $S$ satisfies the three rules for being a partial order on function space $\mathcal{F}(X,P)$:
\begin{enumerate}
    \item (antisymmetry) Since $P$ is a partially ordered set and $f(x),g(x)\in P$, then if $f(x)\leq g(x)$ and $g(x)\leq f(x)$, $f(x)=g(x)$ by the antisymmetry of some partial order on $P$.
    \item (transitivity) If $f(x)\leq g(x)$ and $g(x)\leq h(x)$, then $f(x)\leq h(x)$ by the transitivity of some partial order on $P$. Therefore $(f,h)\in S$ as well.
    \item (reflexivity) For all functions $f$ in the function space, $(f,f)\in S$ since $f(x)=f(x)$ implies $f(x)\leq f(x)$.
\end{enumerate}
Thus, $S$ is a partial order on the function space $\mathcal{F}(X,P)$ determined by $f\leq g$ if $f(x)\leq g(x)$.

\textbf{1.3 Solution.} The Identity function takes $X$ back to itself, so $P(ID_X)=P(X)$. It also takes $P(X)$ back to itself, $P(X)$. Therefore $P(ID_X)=P(X)=ID_{P(X)}$. 

$P(g\circ f)$ takes a subset $S_X\subseteq X$ and sends it to a subset $S_Z\subseteq Z$, whereas $P(g)\circ P(f)$ takes the set $S_X$ to a set $S_Y\subseteq Y$ and then takes $S_Y$ to another set $T_Z\in Z$. For every $x\in S_X$, $P(g\circ f)$ sends $x$ to a $z=g(f(x))\in S_Z$. On the other hand, $P(g)\circ P(f)$ first sends $x$ to a $y=f(x)\in S_Y$, and then sends $y$ to $z=g(y)=g(f(x))\in T_Z$. But since both the LHS and RHS send $x\in S_X$ to $g(f(x))$, it must be true that $S_Z=T_Z$, and therefore the equality $P(g\circ f)=P(g)\circ P(f)$ holds.

\textbf{1.4 Solution.} (a) \boxed{\text{False.}} We provide a counter example. Let $S=\{1\}$, $X=\{1, 2\}$, $Y=\{3\}$, and $f(1)=f(2)=3$. Then $f(S)=\{3\}$, and $f^{-1}(\{3\})=\{1, 2\}$ since the preimage of $\{3\}$ includes the element 2 in addition to 1. Therefore $f^{-1}(f(S))\neq S$.

(b) Define function $f:X\to Y$. Since it is invertible, we have $f^{-1}:Y\to X$ s.t. every $x\in X$ maps to a unique $y\in Y$ and vice versa. In other words, $f$ is bijective. 
